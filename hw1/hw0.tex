% ---------
%  Compile with "pdflatex hw0".
% --------
%!TEX TS-program = pdflatex

\documentclass[11pt]{article}
\usepackage{jeffe,handout,graphicx}

%!TEX encoding = UTF-8 Unicode
\usepackage[utf8]{inputenc}		% Allow some non-ASCII Unicode in source

% ---------
%  The next following lines change the default text and math fonts and
%  make a few other minor cosmetic changes.  If you get any error
%  messages related to these packages, just comment them out.  -- Jeff
% --------
\usepackage[charter]{mathdesign}
\def\sfdefault{fvs}
\def\ttdefault{fvm}
\SetMathAlphabet{\mathsf}{bold}{\encodingdefault}{\sfdefault}{b}{\updefault}
\SetMathAlphabet{\mathtt}{bold}{\encodingdefault}{\ttdefault}{b}{\updefault}
\SetMathAlphabet{\mathsf}{normal}{\encodingdefault}{\sfdefault}{\mddefault}{\updefault}
\SetMathAlphabet{\mathtt}{normal}{\encodingdefault}{\ttdefault}{\mddefault}{\updefault}
\usepackage{microtype}
% ---------
%  End of cosmetics.
% --------

% ---------
%  Redefine suits
% --------
\usepackage{pifont}
\def\Spade{\text{\textcolor{Black}{\ding{171}}}}
\def\Heart{\text{\textcolor{Red}{\ding{170}}}}
\def\Diamond{\text{\textcolor{Red}{\ding{169}}}}
\def\Club{\text{\textcolor{Black}{\ding{168}}}}

% =========================================================
\begin{document}

\headers{CS 473 Homework 0}{}{Michele Esposito (mresposit5)}
\small\sf
\begin{enumerate}
\parindent 1.5em \itemsep 0.5in plus 0.25fil

%----------------------------------------------------------------------
\newpage
\def\arraystretch{1.2}

\item
The Lucas numbers $L_n$ and Anti-Lucas numbers $\Gamma_n$ are defined recursively as follows:
\[
	L_n = \begin{cases}
		2 & \text{if $n=0$}\\
		1 & \text{if $n=1$}\\
		L_{n-2} + L_{n-1} & \text{otherwise}
	\end{cases}
	\qquad\qquad
		\Gamma_n = \begin{cases}
		1 & \text{if $n=0$}\\
		2 & \text{if $n=1$}\\
		\Gamma_{n-2} - \Gamma_{n-1} & \text{otherwise}
	\end{cases}
\]

\begin{enumerate}
\item
Prove that $\Gamma_n = (-1)^{n-1} L_{n-1}$ for every positive integer $n$.

\begin{solution}[induction]
Let $n$ be an arbitrary non-negative integer.  There are several cases to consider:
\begin{itemize}
\item
Blah

\item
Snort
\begin{itemize}
\item
Squee

\item
Flub
\end{itemize}

\item
Kronk
\end{itemize}
In all cases, $\Gamma_n = (-1)^{n-1} L_{n-1}$, as required.
\end{solution}

\bigskip

\item
Prove that any non-negative integer can be written as the sum of distinct \emph{non-consecutive} Lucas numbers.

\begin{solution}
This result follows immediately from Flobbersnort’s Fundamental Theorem of anti-dimensional motivic $k$-schemes, which is in turn an obvious consequence of  Flibbertygibbet’s Cocohohomomolology Lemma, as described on page 6 of Jeff’s recurrences notes.
\end{solution}

\end{enumerate}


%----------------------------------------------------------------------
\newpage
\item
Consider the language generated by the context-free grammar
$
	S \to
	\Heart \mid
	\Spade S \mid
	S \Club \mid
	S \Diamond S.
$
Prove that every string in this language has the following properties:
\begin{enumerate}
\item 
The number of $\Heart$s is exactly one more than the number of $\Diamond$s.

\begin{solution}
Sed ut perspiciatis, unde omnis iste natus error sit voluptatem \Spade\ accusantium doloremque laudantium, totam rem aperiam eaque ipsa, quae ab illo inventore veritatis et quasi architecto beatae \Heart\ vitae dicta sunt, explicabo. Nemo enim ipsam voluptatem, quia voluptas \Club\ sit, aspernatur aut odit aut fugit, sed quia consequuntur magni dolores eos, qui ratione voluptatem sequi nesciunt, neque porro quisquam est \Club\, qui do\textbf{lorem ipsum}, quia dolor \Club\ sit amet, consectetur, adipisci velit, sed quia non numquam eius modi tempora \Diamond\ incidunt, ut labore et dolore magnam aliquam quaerat voluptatem. Ut enim ad minima veniam, quis nostrum \Spade\ exercitationem ullam corporis suscipit laboriosam, nisi ut aliquid ex ea commodi \Spade\ consequatur? Quis autem vel eum iure reprehenderit, qui in ea voluptate velit esse, quam nihil \Heart\ molestiae \Club\ consequatur, vel illum, qui dolorem eum fugiat \Diamond\, quo voluptas nulla \Heart\ pariatur \Club?

\begin{flushright}
--- Marcus Tullius Cicero, \emph{De finibus bonorum et malorum} (45 BCE)
\end{flushright}
\end{solution}

\bigskip
\item 
There is a $\Diamond$ between any two $\Heart$s.

\begin{solution}
“Now, why might that be unfair?” [Ricky Jay] continued. “I’ll tell you why. Because, even though you shuffled, I dealt the cards. That time, I also shuffled the cards. Now, this time you shuffle the cards and you deal the cards. And you pick the number of players. And you designate any hand for me and any hand for you.” After shuffling, I dealt four hands, arranged as the points of a square. I chose a hand for myself and selected one for him. My cards added up to nothing---king-high nothing.

“Is that fair?” Jay said, picking up his cards, waiting a beat, and returning them to the table, one by one---the coup de grace. “I. Don’t. Think. So.” One, two, three, four aces.

\begin{flushright}
--- Marc Singer, “Secrets of the Magus”, \emph{New Yorker} (April 5, 1993)
\end{flushright}
\end{solution}

\end{enumerate}

%----------------------------------------------------------------------
\def\BOX{\mathbin\square}
\newpage
\item
Given two undirected graphs $G = (V,E)$ and $G' = (V',E')$, the \emph{box product} $G\BOX G'$ is defined as follows:
\begin{itemize}
\item 
The vertices of $G\BOX G'$ are all pairs $(v, v')$ where $v\in V$ and $v'\in V'$.
\item
Two vertices $(v,v')$ and $(w,w')$ of $G\BOX G'$ are connected by an edge in $G\times G'$ if and only if either \big($v = w$ and $(v',w')\in E'$\big) or \big($(v,w)\in E$ and $v' = w'$\big).
\end{itemize}

\begin{enumerate}
\item
Let $I$ denote the unique connected graph with two vertices.  Give a concise \emph{English} description of the following graphs.
\begin{enumerate} 
\item What is $I\BOX I$?
\item What is $I\BOX I\BOX I$?
\item What is $I\BOX I\BOX I\BOX I$?
\end{enumerate}
\begin{solution}~
\begin{enumerate} 
\item A man with three buttocks.
\item A man with two noses.
\item \strike{A man with nine legs.} A Scotsman on a horse.
%\item A man with a tape recorder up his nose.
\end{enumerate}
\vspace{-3.5ex}
\end{solution}

\bigskip
\item
Recall that a Hamiltonian path in a graph $G$ is a path in $G$ that visits every vertex of~$G$ exactly once.  Prove that for any graphs $G$ and $G'$ that both contain Hamiltonian paths, the box product $G\BOX G'$ also contains a Hamiltonian path.

\begin{solution}~
\begin{description}
\item
\textsc{Smart Cookie:} Wouldn't it have been easier to use the door, Chancellor?
\item
\textsc{Chancelor Puddinghead:} Maybe for you, Smart Cookie. But I am a chancellor. I was elected because I know how to think outside the box. Which means \emph{[goes into the fireplace]} I~can also think inside the chimney! Can you think inside a chimney?
\item
\textsc{Smart Cookie:} I\dots
\item
\textsc{Chancelor Puddinghead:} \emph{[walks by with coal soot covering her face]} I didn't think so!
\end{description}

\begin{center}
\includegraphics[height=2in]{Fig/puddinghead}
\end{center}

\begin{flushright}
--- “Hearth’s Warming Eve”, \emph{My Little Pony: Friendship is Magic} (December 17, 2011)
\end{flushright}
\end{solution}



\end{enumerate}

%----------------------------------------------------------------------
\newpage
\item
Describe and analyze a data structure that stores a set $S$ of $n$ points in the plane, each represented by a pair of integer coordinates, and supports queries of the following form:
\begin{quote}
\begin{description}
\item $\textsc{\textrm{SomethingAboveRight}}(x,y)$:
Return an arbitrary point $(a,b)$ in $S$ such that $a>x$ and $b>y$.  If there is no such point in $S$, return \textsc{\textrm{None}}.
\end{description}
\end{quote}
(a) Describe your data structure; (b) analyze the space it uses; (c) describe your query algorithm; (d) prove that it is correct; and (e) analyze its worst-case running time.

\begin{solution}~
\begin{enumerate}
\item
There was an old man of St. Bees,

\item
Who was stung in the arm by a wasp;

\item
When they asked, “Does it hurt?”

\item
He replied, “No, it does n’t,

\item 
But I thought all the while 't was a hornet!”
\end{enumerate}

\begin{flushright}
--- W. S. Gilbert, “A Nonsense Rhyme in Blank Verse”\\
quoted in “Pigeonhole Paragraphs”, \emph{The Irish Review} (November 1898)
\end{flushright}
\end{solution}



%----------------------------------------------------------------------
\newpage
\Hard
\item \EMPH{[Extra credit]}
Prove that any Gaussian integer can be expressed as the sum of distinct powers of the complex number $\alpha = -1+i$.

\begin{solution}
The imaginary number is a fine and wonderful resource of the divine spirit, as if it were an amphibian of existence and non-existence.
\begin{flushright}
--- Gottfried Wilhelm Leibniz (1646--1716)
\end{flushright}

\medskip
\noindent
That this subject [imaginary numbers] has hitherto been surrounded by mysterious obscurity, is to be attributed largely to an ill adapted notation. If, for example, $+1$, $-1$, and the square root of $-1$ had been called direct, inverse and lateral units, instead of positive, negative and imaginary (or even impossible), such an obscurity would have been out of the question.
\begin{flushright}
--- Carl Friedrich Gauss (1777--1855)
\end{flushright}

\medskip
\noindent
\begin{description}
\item \textsc{Professor:} So, as you can see, the square root of negative one is the imaginary number I.
\item \emph{[A student's head explodes]}
\item \emph{[A long pause]}
\item \textsc{Student:} I don't get it. \emph{[pause]}  Oh!  Now I get--- \emph{[head explodes]}
\end{description}
\begin{flushright}
--- “Unionizing Our Labor”, \emph{Robot Chicken} (August 9, 2009)
\end{flushright}
\end{solution}


%----------------------------------------------------------------------

\end{enumerate}

\end{document}
